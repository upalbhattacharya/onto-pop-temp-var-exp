\documentclass[a4paper,colorinlistoftodos]{article}
%%%%%%%%%%%%%%%%%%%%%%%%%%%%%%%%%%%%%%%%%%%%%%%%%%%%%%%%%%%%%%%%%%%%%%
% Package Loading
%%%%%%%%%%%%%%%%%%%%%%%%%%%%%%%%%%%%%%%%%%%%%%%%%%%%%%%%%%%%%%%%%%%%%%

%%%%%%%%%% Basic %%%%%%%%%%
\usepackage[utf8]{inputenc}
\usepackage[T1]{fontenc}
\usepackage[displaymath,mathlines,running]{lineno}

%%%%%%%%%% Referencing %%%%%%%%%%
\usepackage[numbers,sort&compress]{natbib}

%%%%%%%%%% Images %%%%%%%%%%
\usepackage{graphicx}
\usepackage{subcaption}      % For subfigures

%%%%%%%%%% Formatting %%%%%%%%%%
\usepackage{longtable}
\usepackage[hidelinks]{hyperref}
\hypersetup{
  colorlinks = true,
  linkcolor = cat-latte-red}

%%%%%%%%%% Math %%%%%%%%%%
\usepackage{amsmath}
\usepackage{amssymb}
\usepackage{amsthm}

%%%%%%%%%% Presentation %%%%%%%%%%
\usepackage[dvipsnames]{xcolor}
\usepackage[normalem]{ulem}
\usepackage{tcolorbox}
\usepackage{soul}


%%%%%%%%%% Misc %%%%%%%%%%
\usepackage{todonotes}

\newcounter{todocounter}
\newcommand{\todonum}[2][]
{\stepcounter{todocounter}\todo[#1]{\thetodocounter: #2}}

\usepackage[commandnameprefix=always]{changes}
\definechangesauthor[name='Upal Bhattacharya', color=cat-latte-purple]{UB}

%%%%%%%%%%%%%%%%%%%%%%%%%%%%%%%%%%%%%%%%%%%%%%%%%%%%%%%%%%%%%%%%%%%%%%
% Customization
%%%%%%%%%%%%%%%%%%%%%%%%%%%%%%%%%%%%%%%%%%%%%%%%%%%%%%%%%%%%%%%%%%%%%%
\tcbuselibrary{theorems}

%%%%%%%%%% Colors (Catppuccin Latte) %%%%%%%%%%
\definecolor{cat-latte-green}{HTML}{40A02B}
\definecolor{cat-latte-orange}{HTML}{FE640B}
\definecolor{cat-latte-yellow}{HTML}{DF8E1D}
\definecolor{cat-latte-red}{HTML}{D20F39}
\definecolor{cat-latte-blue}{HTML}{04A5E5}
\definecolor{cat-latte-gray}{HTML}{7c7f93}
\definecolor{cat-latte-purple}{HTML}{8839EF}
\definecolor{cat-latte-lavendar}{HTML}{7287fd}

\colorlet{soulred}{cat-latte-red!50}
\colorlet{soulgreen}{cat-latte-green!50}
\colorlet{soulyellow}{cat-latte-yellow!50}
\colorlet{soulorange}{cat-latte-orange!50}
\colorlet{soulblue}{cat-latte-blue!50}
\colorlet{soulgray}{cat-latte-gray!50}

%%%%%%%%%% Callouts %%%%%%%%%%
\newtcbtheorem[auto counter]{caution}{Caution}%
{colback=cat-latte-orange!15,colframe=cat-latte-orange,fonttitle=\bfseries}{caution}
\newtcbtheorem[auto counter]{positive}{Positive}%
{colback=cat-latte-green!15,colframe=cat-latte-green,fonttitle=\bfseries}{positive}
\newtcbtheorem[auto counter]{query}{Query}%
{colback=cat-latte-yellow!15,colframe=cat-latte-yellow,fonttitle=\bfseries}{query}
\newtcbtheorem[auto counter]{answer}{Answer}%
{colback=cat-latte-blue!15,colframe=cat-latte-blue,fonttitle=\bfseries}{answer}
\newtcbtheorem[auto counter]{negative}{Negative}%
{colback=cat-latte-red!15,colframe=cat-latte-red,fonttitle=\bfseries}{negative}
\newtcbtheorem[auto counter]{gap}{Gap}%
{colback=cat-latte-gray!15,colframe=cat-latte-gray,fonttitle=\bfseries}{gap}
\newtcbtheorem[no counter]{upal}{Upal}%
{colback=cat-latte-purple!15,colframe=cat-latte-purple,fonttitle=\bfseries}{upal}
\newtcbtheorem[no counter]{project}{Project}%
{colback=cat-latte-lavendar!15,colframe=cat-latte-lavendar,fonttitle=\bfseries}{project}

%%%%%%%%%% Highlights %%%%%%%%%%
\newcommand{\hlred}[1]{\sethlcolor{soulred}\hl{#1}}
\newcommand{\hlgreen}[1]{\sethlcolor{soulgreen}\hl{#1}}
\newcommand{\hlyellow}[1]{\sethlcolor{soulyellow}\hl{#1}}
\newcommand{\hlblue}[1]{\sethlcolor{soulblue}\hl{#1}}
\newcommand{\hlorange}[1]{\sethlcolor{soulorange}\hl{#1}}
\newcommand{\hlgray}[1]{\sethlcolor{soulgray}\hl{#1}}

%%%%%%%%%% Underline %%%%%%%%%%
\setul{}{1pt} % Width
\newcommand{\ulred}[1]{\setulcolor{cat-latte-red}\ul{#1}}
\newcommand{\ulgreen}[1]{\setulcolor{cat-latte-green}\ul{#1}}
\newcommand{\ulyellow}[1]{\setulcolor{cat-latte-yellow}\ul{#1}}
\newcommand{\ulblue}[1]{\setulcolor{cat-latte-blue}\ul{#1}}
\newcommand{\ulorange}[1]{\setulcolor{cat-latte-orange}\ul{#1}}
\newcommand{\ulgray}[1]{\setulcolor{cat-latte-gray}\ul{#1}}

%%%%%%%%%%%%%%%%%%%%%%%%%%%%%%%%%%%%%%%%%%%%%%%%%%%%%%%%%%%%%%%%%%%%%%
% Project-Specific
%%%%%%%%%%%%%%%%%%%%%%%%%%%%%%%%%%%%%%%%%%%%%%%%%%%%%%%%%%%%%%%%%%%%%%

% Place project-specific imports here

%%%%%%%%%% Imports and Commands %%%%%%%%%%
\newtcbtheorem[auto counter]{hypothesis}{Hypothesis}%
{colback=cat-frappe-sky!15,colframe=cat-frappe-sky,fonttitle=\bfseries}{hypothesis}

%%%%%%%%%% Customization %%%%%%%%%%

\definecolor{cat-frappe-sky}{HTML}{99d1db}
 % For common headers, utilities and styling
% \usepackage{environ}

\newif\ifhide
\hidetrue % toggle if necessary

% Callouts
\ifhide
  \NewEnviron{hide}{}
  \let\caution\hide
  \let\endcaution\endhide
  
  \let\positive\hide
  \let\endpositive\endhide
  
  \let\negative\hide
  \let\endnegative\endhide
  
  \let\gap\hide
  \let\endgap\endhide
  
  \let\upal\hide
  \let\endupal\endhide
  
  \let\query\hide
  \let\endquery\endhide
  
  \let\answer\hide
  \let\endanswer\endhide
  
  \let\project\hide
  \let\endproject\endhide
\fi

% Highlights
\renewcommand\hlred[1]{#1}
\renewcommand\hlgreen[1]{#1}
\renewcommand\hlblue[1]{#1}
\renewcommand\hlyellow[1]{#1}
\renewcommand\hlorange[1]{#1}
\renewcommand\hlgray[1]{#1}

% Underline
\renewcommand\ulred[1]{#1}
\renewcommand\ulgreen[1]{#1}
\renewcommand\ulblue[1]{#1}
\renewcommand\ulyellow[1]{#1}
\renewcommand\ulorange[1]{#1}
\renewcommand\ulgray[1]{#1}

% To hide todonotes, use: \usepackage[disable]{todonotes} in preamble.tex
% To hide changes, use: \usepackage[final]{changes} in preamble.tex
 % For hiding TODOS, highlights, underlines and callouts

% Add other dependencies in preamble.tex (for cleaner visuals)

\author{Upal Bhattacharya}
\date{\today}
\title{Comprehensive LLM Response Variability over Temperature Variation}
\begin{document}

\maketitle

\begingroup
    \hypersetup{linkcolor=black}
    \tableofcontents
    \listoftodos
    \pagebreak
\endgroup
\linenumbers
%%%%%%%%%%%%%%%%%%%%%%%%%%%%%%%%%%%%%%%%%%%%%%%%%%%%%%%%%%%%%%%%%%%%%%
% BEGIN CONTENT
%%%%%%%%%%%%%%%%%%%%%%%%%%%%%%%%%%%%%%%%%%%%%%%%%%%%%%%%%%%%%%%%%%%%%%
\abstract{The temperature hyperparameter ($T \geq 0$) of LLMs facilitates
annealing of the conditional token probabilities and drives output
determinism. For any LLM-driven ontology learning task (in this situation:
ontology population), understanding the degree of variability introduced by
LLMs performing ontology population over a spectrum of temperature values
helps to pin-point value ranges that provide the best $mAP@D$ and the lowest
variability range. \textit{Prior experimentation with three temperature values
(low, medium and high) on two ontologies (Wines
\cite{noy2001OntologyDevelopment101} and CASE
\cite{casey2018EvolutionExpressingExchanging}) and with two LLMs (GPT-4o
\cite{openai2024HelloGpt4o} and Llama3-8B \cite{grattafiori2024Llama3Herd})
observed that varying the temperature value did not lead to large differences
in performance variability or average performance. The larger LLM (GPT-4o)
exhibited lower variability (statistically insignificant) while the smaller
LLM (Llama3-8B) had much larger variability (statistically significant). The
temperature study did not observe performance variability differences between
the two ontologies either.}}

\section{Previous Experimentation}
\label{sec:prev-temp-variation}

A preliminary set of experiments (for ISWC 2025) was carried out to gauge the
degree of response variation produced by different temperature settings. The
following sections outline the previous methodology, observations and
limitations.

\subsection{Method}
\label{subsec:temp-variation-prev-method}

The effect of temperature variability on performance consistency was measured
over two ontologies: Wines \cite{noy2001OntologyDevelopment101} and CASE
\cite{casey2018EvolutionExpressingExchanging}. The selection was based on the
structural differences between the two ontologies. Experiments were performed
on GPT-4o and Llama3-8B using ontology domain contextualization and 3-shot
prompting.

To analyze the effect of temperature, three temperature settings corresponding
to 0.2 (low), 0.5/0.6 (default; the default value is close to the mid-point
over the range of temperature values but varies between LLMs) and 0.8 (high)
were selected. At each temperature value, experiments were repeated 10 times
and the range and average $\text{mAP@D}$ values were calculated.

\subsection{Observations}
\label{subsec:temp-variation-prev-observations}

\begin{figure}
\begin{subfigure}{0.5\textwidth} \centering
    \caption{Wines Ontology}
    \label{fig:prev-wines-temp-variation}
\includegraphics[scale=0.32]{assets/wines-ontology-temp-stat-variation-ontology.pdf}
\end{subfigure} \hspace{0.2cm}
\begin{subfigure}{0.5\textwidth} \centering
    \caption{CASE Ontology}
    \label{fig:prev-case-temp-variation}
\includegraphics[scale=0.32]{assets/case-uco-owl-trafficking-temp-stat-variation-ontology.pdf}
\end{subfigure}
\caption{Temperature variation for (a) Wines and (b) CASE ontologies for
GPT-4o and Llama3-8B with 3-shot, ontology domain contextualization. Central
points are the average value over 10 runs. Lighter shaded regions are
approximated densities over the range of values}
\label{fig:prev-temp-variation}
\end{figure}

Figure \ref{fig:prev-temp-variation} highlights the differences in range,
average values and approximate density over 10 repetitions for both ontologies
(Wines in Figure \ref{fig:prev-wines-temp-variation}; CASE in Figure
\ref{fig:prev-case-temp-variation}) and LLMs.

GPT-4o was fairly consistent with very small ranges of $\text{mAP@D}$ observed
over all temperature values and across both ontologies. The effect of response
variability on $\text{mAP@D}$ was more pronounced in Llama3-8B. Contrary to
expectation, lower temperature values did not result in less variability. The
low temperature setting did not result in higher average $\text{mAP@D}$ than
the default value for both LLMs and across both ontologies. The high
temperature setting led to a decline in performance over the default
value. This decrease was statistically significant for GPT-4o but the large
range of variability rendered the same trend non-significant for Llama3-8B.

The main takeaway from the experimentation was that the effect of temperature
variation was not strongly pronounced. The hypothesis that higher temperature
values contributed to greater variability and therefore exhibited more
variable $\text{mAP@D}$ could not be validated. Performance appeared to worsen
when using a high temperature value.

\subsection{Limitations}
\label{subsec:temp-variation-prev-limitations}

The experimentation for temperature variation provides several observations
but are not comprehensive enough to be considered strong statements about
reliable trends.

% Considered number of temperature values very small The theoretical
temperature range ($T \geq 0$) is a large, continuous range of infinite values
to be considered. While certain ranges certainly correspond to expected LLM
behaviour, considering only singular values from these `zones' is insufficient
to gain a robust understanding of observed behaviour. Considering more values
from the `low', `medium' and `high' value ranges provides a better
fine-grained breakdown of temperature variation.

% Does not explore variability at T=0 and T=1 (or max.)  Performance observed
at the theoretical limits ($T=0$; $T=\infty$) is not explored. The degree of
variability observed when the model is defined as `completely deterministic'
acts as a control for determinism but is lacking in the present
experimentation. The same is true for the upper end of the value range (in
practice, different models have an upper bound of their temperature, which can
be construed as the equivalent of the theoretical value for greatest response
variability).

% No in-depth analysis An analysis of the number of concepts predicted at
different temperature ranges or the consistency of prediction is
missing. Macro-metrics provide a sense of the average performance but
variability is more pronounced and understood at a per-instance
scenario. Other methods of analyzing the responses that examine trends in
predicted concepts would provide a realistic and interpretable method of
assessing response variability with temperature.

\section{New Experimentation}
\label{sec:new-temp-variation}

Based on the limitations outlined in Section
\ref{subsec:temp-variation-prev-limitations}, a more comprehensive set of
experiments and approaches is to be performed to analyze the variability that
arises from temperature variation.

% Insufficient number of repetitions for good reliability

\subsection{Method}
\label{subsec:new-temp-variation-method}

% Experimental variables %% 2 ontologies %% 2 LLMs %% 10 repetitions

In terms of the experimental setup, the same set of LLMs (GPT-4o and
Llama3-8B) and ontologies (Wines and CASE) are to be used with ontology domain
contextualization and using 3-shot prompting with 10 repetitions for each
combination. The primary change is in considering more values in the different
`zones'. In particular, we consider the LLM-equivalent temperature values $T
\in \{0.0,\ 0.1,\ 0.2,\ 0.3,\ 0.4,\ 0.5,\ 0.6,\ 0.7,\ 0.8,\ 0.9,\ 1.0\}$ to
analyze performance. The larger set of values incorporates edge conditions
i.e. $0.0$ and $1.0$ as well as more values in the `low', `medium' and `high'
ranges (low: $\{0.1, 0.2, 0.3\}$; medium: $\{0.4, 0.5, 0.6\}$; high: $\{0.7,
0.8, 0.9\}$).

In addition to more dense experimentation to understand the macroscopic
behaviour differences (i.e. average $\text{mAP@D}$ and range of
$\text{mAP@D}$), analyzing each response instance provides a more accurate
understanding of the nature of variability.

The following hypotheses apply a more nuanced strategy to investigate the
exact nature of variability introduced by temperature variation.

\begin{hypothesis}
    \label{hyp:hyp-1} The number of directly-asserted concepts predicted by an
LLM for ontology population increases with increase in temperature.
\end{hypothesis}

\begin{hypothesis}
    \label{hyp:hyp-2} The distribution of directly-asserted concepts predicted
by an LLM for ontology population becomes more `platykurtic' with increase in
temperature.
\end{hypothesis}

\begin{hypothesis} The frequency with which an LLM predicts directly-asserted
concepts for ontology population from the contextual examples provided
increases with decrease in temperature.
\end{hypothesis}

\begin{hypothesis} The consistency of prediction for ontology population for
an individual by an LLM decreases with increase in temperature.
\end{hypothesis}

\begin{hypothesis} The frequency of predicting out-of-context or non-existent
concepts by an LLM for ontology population increases with increase in
temperature.
\end{hypothesis}

The above hypotheses focus on the directly-asserted concepts for
individuals. However, they are equally applicable for the rest of the
hierarchy (or ranked lists) generated by the LLMs.

\bibliographystyle{splncs04nat}
\bibliography{bibliography}
\end{document}

